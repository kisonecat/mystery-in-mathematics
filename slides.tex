\documentclass[14pt,aspectratio=169]{beamer}

\author{Jim Fowler (\texttt{@kisonecat})\\ Department of Mathematics \\ The Ohio State University}
\date{January 13, 2016}
\title{Mystery in Mathematics: \\ Diffie-Hellman Key Exchange}

\definecolor{ccgray}{HTML}{a7b1a6}


\usepackage{xcolor}
\usepackage{pgf}
\usepackage{amsmath}
\usepackage{amssymb}
\usepackage{latexsym}
\usepackage{tikz}
\usepackage{pgfplots}
\usepackage{pdfpages}


\makeatletter
\define@key{beamerframe}{nofills}[true]{% top
  \beamer@frametopskip=0pt\relax%
  \beamer@framebottomskip=0pt\relax%
  \beamer@frametopskipautobreak=\beamer@frametopskip\relax%
  \beamer@framebottomskipautobreak=\beamer@framebottomskip\relax%
  \def\beamer@initfirstlineunskip{%
    \def\beamer@firstlineitemizeunskip{%
      \vskip-\partopsep\vskip-\topsep\vskip-\parskip%
      \global\let\beamer@firstlineitemizeunskip=\relax}%
    \everypar{\global\let\beamer@firstlineitemizeunskip=\relax}}
}
\makeatother

\setbeamertemplate{navigation symbols}{}

\begin{document}

\begin{frame}
\maketitle
\end{frame}

\begin{frame}
  \Huge

  \begin{center}
    Mystery of Mathematics
  \end{center}
\end{frame}

\begin{frame}
\hangindent=0.7cm Wigner, Eugene P. ``The unreasonable effectiveness of mathematics in the natural sciences.'' \textit{Communications on pure and applied mathematics} 13, no. 1 (1960): 1--14.

\vfill

\hangindent=0.7cm Putnam, Hilary. ``What is mathematical truth?'' \textit{Historia Mathematica} 2, no. 4 (1975): 529--533.

\vfill

\hangindent=0.7cm Cheng, Eugenia. ``Mathematics, morally.'' (2004).
\end{frame}

\begin{frame}
  \Huge

  \begin{center}
    Mathematics of Mystery 
  \end{center}
\end{frame}

%%%%%%%%%%%%%%%%%%%%%%%%%%%%%%%%%%%%%%%%%%%%%%%%%%%%%%%%%%%%%%%%

\begin{frame}
  \huge

  Can two friends, Alice and Bob, \\
  \quad communicate in public \\
  \quad \quad (with eavesdroppers!) \\
  and nevertheless \\
  \quad come to agree on a secret \\
  \quad only they share?
\end{frame}

\newcommand{\swatch}[1]{\textcolor{#1}{\rule{12pt}{12pt}}}

\newcommand{\aliceroombob}[3]{%
  \begin{tabular}{@{}p{0.33\textwidth}@{}p{0.33\textwidth}@{}p{0.33\textwidth}@{}}
    \begin{center}#1\end{center} & \begin{center}#2\end{center} & \begin{center}#3\end{center} 
  \end{tabular}\\[-6ex]}

\newcommand{\alice}[1]{\aliceroombob{#1}{}{}}
\newcommand{\alicebob}[2]{\aliceroombob{#1}{}{#2}}
\newcommand{\room}[1]{\aliceroombob{}{#1}{}}
\newcommand{\bob}[1]{\aliceroombob{}{}{#1}}


\begin{frame}[nofills]
  \aliceroombob
  {\textbf{Alice's Mind}}
  {\textbf{The Room}}
  {\textbf{Bob's Mind}}
  
  \uncover<2->{\room{``Start with \swatch{yellow}.''}}
  \alicebob{\uncover<3->{Pick \swatch{blue!75!white}}}{\uncover<4->{Pick \swatch{red}}}
  \alicebob
  {\uncover<5->{Mix \swatch{yellow} and \swatch{blue!75!white} = \swatch{green!75!black}}}
  {\uncover<7->{Mix \swatch{yellow} and \swatch{red} = \swatch{orange}}}
  \uncover<6->{\room{Alice says, ``\,\swatch{green!75!black}.''}}
  \uncover<8->{\room{Bob says, ``\,\swatch{orange}.''}}
  \alicebob{\uncover<9->{Mix \swatch{orange} and \swatch{blue!75!white} = \swatch{brown}}}
  {\uncover<10->{Mix \swatch{green!75!black} and \swatch{red} = \swatch{brown}}}
  \uncover<11->{\alicebob{Our secret is \swatch{brown}}{Our secret is \swatch{brown}}}

\end{frame}

\newcommand{\snumber}[2]{\colorbox{#1}{\textcolor{white}{#2}}}
\newcommand{\lnumber}[2]{\colorbox{#1}{\textcolor{black}{#2}}}

\begin{frame}[nofills]
  \aliceroombob
  {\textbf{Alice's Mind}}
  {\textbf{The Room}}
  {\textbf{Bob's Mind}}
  
  \uncover<2->{\room{``Start with \lnumber{yellow}{2}.''}}
  \alicebob{\uncover<3->{Pick \snumber{blue!75!white}{24}.}}{\uncover<4->{Pick \snumber{red}{17}.}}
  \alicebob
  {\uncover<5->{$M(\lnumber{yellow}{2}, \snumber{blue!75!white}{24}) = \snumber{green!75!black}{20}$.}}
  {\uncover<7->{$M(\lnumber{yellow}{2}, \snumber{red}{17}) = \snumber{orange}{21}$.}}
  \uncover<6->{\room{Alice says, ``\,\snumber{green!75!black}{20}.''}}
  \uncover<8->{\room{Bob says, ``\,\snumber{orange}{21}.''}}
  \alicebob{\uncover<9->{$M(\snumber{orange}{21}, \snumber{blue!75!white}{24}) = \snumber{brown}{25}$.}}
  {\uncover<10->{$M(\snumber{green!75!black}{20}, \snumber{red}{17}) = \snumber{brown}{25}$.}}
  \uncover<11->{\alicebob{Our secret is \snumber{brown}{25}.}{Our secret is \snumber{brown}{25}.}}

\end{frame}

\begin{frame}
  \frametitle{Why does this work?}

  \Large

  In this case, \[M(M(\lnumber{yellow}{2}, \snumber{blue!75!white}{24}), \snumber{red}{17}) =
  M(M(\lnumber{yellow}{2}, \snumber{red}{17}), \snumber{blue!75!white}{24}).\]

\vfill\pause

  In general, \[M(M(\lnumber{yellow}{2},a),b) = M(M(\lnumber{yellow}{2},b),a).\]

\vfill\pause

  Why does ``Why does this work?'' work?

\end{frame}

%%%%%%%%%%%%%%%%%%%%%%%%%%%%%%%%%%%%%%%%%%%%%%%%%%%%%%%%%%%%%%%%
% THANK YOU
 \begin{frame}[label=thanks,nofills]
   \vfill
   \begin{center}
   \Huge
    \scalebox{1.5}{\textbf{Thank You}}

    \texttt{http://go.osu.edu/mystery}
   \end{center}
   \vfill
   \vfill
   \includegraphics[width=1in]{images/cc-logo.pdf}\hfill\footnotesize\scalebox{0.75}{\textcolor{ccgray}{Licensed for reuse under a Creative Commons BY-SA License}}
   \null
   %\vspace{12pt}
   \null
 \end{frame}


\end{document}

%%% Local Variables:
%%% mode: latex
%%% TeX-master: t
%%% End:
